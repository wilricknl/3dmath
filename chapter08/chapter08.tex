\documentclass[11pt]{article}

\author{wilricknl}
\title{3D Math Primer - Solutions chapter 8}

\usepackage{amsmath}
\usepackage{amssymb}
\usepackage{float}
\usepackage{enumerate}
\usepackage{graphicx}
\usepackage{url}

\begin{document}

\maketitle

\section{Solutions chapter 8}

\subsection{Exercise 1}

\begin{enumerate}[a.]
	\item 5
	\item 3
	\item 6
	\item 1
	\item 2
	\item 4
\end{enumerate}

\subsection{Exercise 2}

\begin{table}[H]
\centering
\begin{tabular}{|l|l|}
\hline
\textbf{Image} & \textbf{Canonical?} \\ \hline
3 & Yes \\ \hline % a
4 & Yes \\ \hline % b
5 & No, bank needs to be $0^\circ$ as pitch is $90^\circ$ \\ \hline % c
1 & Yes \\ \hline % d
2 & Yes \\ \hline % e
3 & No, pitch outside limits. \\ \hline % f
5 & Yes \\ \hline % g
2 & No, pitch outside limits. \\ \hline % h
6 & Yes \\ \hline % i
\end{tabular}
\caption{Exercise 2}
\label{tab:08-exercise-02}
\end{table}

\subsection{Exercise 3}

\begin{enumerate}[a.]
	\item Thirty degrees around the $x$-axis, results in the following Euler angles: $(0^\circ,30^\circ,0^\circ)$. Which converts to 
	$\begin{bmatrix}
		\cos(\frac{30^\circ}{2}) \\
		\begin{pmatrix}
			1\cdot\sin(\frac{30^\circ}{2}) \\
			0 \\
			0
		\end{pmatrix}
	\end{bmatrix}=\begin{bmatrix}
		0.966 \\
		\begin{pmatrix}
			0.259 \\
			0 \\
			0
		\end{pmatrix}
	\end{bmatrix}$.
	\item 1.
	\item $\begin{bmatrix}
		0.966 \\
		\begin{pmatrix}
			0.259 \\
			0 \\
			0
		\end{pmatrix}
	\end{bmatrix}^*=\begin{bmatrix}
		0.966 \\
		\begin{pmatrix}
			-0.259 \\
			0 \\
			0
		\end{pmatrix}
	\end{bmatrix}$.
	\item Pitch $+30^\circ$.
\end{enumerate}

\subsection{Exercise 4}

\begin{enumerate}[a.]
	\item % a
	$\begin{bmatrix}
		1 & 0 & 0 \\
		0 & 1 & 0 \\
		0 & 0 & 1
	\end{bmatrix}$, so image 2.
	\item % b
		In this case I convert the quaternion to a matrix rotation and compare the result with the matrices from exercise 1. First of all, I compute some common values for the conversion matrix (section 8.7.3 in the book):	
	
		\begin{align*}
			xx&=0.426  &yz&=yy=0.073 \\
			xy&=0.177  &yw&=-xy=-0.177 \\
			xz&=0.177  &zw&=-xz=-0.177 \\
			xw&=-0.426 &zz&=0.073 \\
			yy&=0.073  &ww&=0.426
		\end{align*}

		Then filling in for each matrix element:
		
		\begin{align*}
			m_{11}&=1-0.146-0.146&=&0.708 \\
			m_{12}&=2\cdot(0.177-0.177)&=&0 \\
			m_{13}&=2\cdot(0.177--0.177)&=&0.708 \\
			m_{21}&=2\cdot(0.177-0.177)&=&0.708 \\
			m_{22}&=0&=&0 \\
			m_{23}&=2\cdot(0.073-0.426)&=&-0.708 \\
			m_{31}&=2\cdot(0.177-0.177)&=&0 \\
			m_{32}&=2\cdot(0.073--0.426)&=&1 \\
			m_{33}&=1-2\cdot(0.426+0.073)&=&0
		\end{align*}		
		
		Which results in the following matrix: $\begin{bmatrix}
			0.708 & 0 & 0.708 \\
			0.708 & 0 & -0.708 \\
			0 & 1 & 0
		\end{bmatrix}$, which equals the matrix of exercise 1.a, so it represents the rotation of image 5. For the remaining items the same approach can be used.
	\item % c
	Image 1.
	\item % d
	Image 3.
	\item % e
	Image 2.
	\item % f
	Image 1.
	\item % g
	Image 4.
	\item % h
	Image 6.
	\item % i
	Image 3.
\end{enumerate}

\subsection{Exercise 5}

\begin{enumerate}[a.]
	\item % a
		First, we compute $\cos\frac{\theta}{2}=\cos\frac{98.4^\circ}{2}=0.653$ and $\sin\frac{\theta}{2}=\sin\frac{98.4^\circ}{2}=0.756$. Then the quaternion can be created with the given axis: 
		$$\begin{bmatrix}
			0.653 \\
			\begin{pmatrix}
				\sin\frac{98.4^\circ}{2}\cdot -0.803 \\
				\sin\frac{98.4^\circ}{2}\cdot -0.357 \\
				\sin\frac{98.4^\circ}{2}\cdot -0.357
			\end{pmatrix}
		\end{bmatrix}=\begin{bmatrix}
			0.653 \\
			\begin{pmatrix}
				-0.653 \\
				-0.270 \\
				-0.270
			\end{pmatrix}
		\end{bmatrix}$$
		This equals quaternion b from exercise 4, so image 5. The same approach applies to the remaining items.
	\item % b
	Image 2.
	\item % c
	Image 6.
	\item % d
	Image 1.
	\item % e
	Image 3.
	\item % f
	Image 5.
	\item % g
	Image 4.
	\item % h
	Image 2.
	\item % i
	Image 3.
\end{enumerate}

\subsection{Exercise 6}

By representing quaternions as four dimensional complex numbers and applying the rules of 8.10, we need to derive the quaternion multiplication formula: $$\begin{bmatrix}
	w_1w_2-x_1x_2-y_1y_2-z_1z_2 \\
	\begin{pmatrix}
		w_1x_2 + x_1w_2 + y_1z_2 - z_1y_2 \\
		w_1y_2 + y_1w_2 + z_1x_2 - x_1z_2 \\
		w_1z_2 + z_1w_2 + x_1y_2 - y_1x_2
	\end{pmatrix}
\end{bmatrix}$$

Let's first define two complex numbers:

\begin{itemize}
	\item $q_1=w_1+x_1i+y_1j+z_1k$
	\item $q_2=w_2+x_2i+y_2j+z_2k$
\end{itemize}

Then multiplying and simplifying both complex numbers gives the following:

\begin{align*}
	(w_1+x_1i+y_1j+z_1k)(w_2+x_2i+y_2j+z_2k)=
\end{align*}
Multiplication:
\begin{align*}
	w_1w_2 &+ w_1x_2i + w_1y_2j + w_1z_2k + \\
	x_1w_2i &+ x_1x_2i^2 + x_1y_2ij + x_1z_2ik + \\
	y_1w_2j &+ y_1x_2ji + y_1y_2j^2 + y_1z_2jk + \\
	z_1w_2k &+ z_1x_2ki + z_1y_2kj + z_1z_2k^2 =	
\end{align*}
Applying the rules of 8.10:
\begin{align*}
	w_1w_2  &- x_1x_2 - y_1y_2 - z_1z_2 + \\
	w_1x_2i &+ x_1w_2i + y_1z_2i - z_1y_2i + \\
	w_1y_2j &+ y_1w_2j + z_1x_2j - x_1z_2j + \\
	w_1z_2k &+ z_1w_2k + x_1y_2k - y_1x_2k =
\end{align*}
Simplification:
\begin{align*}
	w_1w_2 &- x_1x_2 - y_1y_2 - z_1z_2 + \\
	(w_1x_2 &+ x_1w_2 + y_1z_2 - z_1y_2)i + \\
	(w_1y_2 &+ y_1w_2 + z_1x_2 - x_1z_2)j + \\
	(w_1z_2 &+ z_1w_2 + x_1y_2 - y_1x_2)k
\end{align*}
We now found the multiplication formula.

\subsection{Exercise 7}

First, let's find out the angle. $\arccos(0.965)=15.2^\circ$, so the angle is $2\cdot15.2^\circ$. Then we can compute the axis $\begin{bmatrix}
0.149 / \sin 15.2^\circ & -0.149 / \sin 15.2^\circ  & 0.149 / \sin 15.2^\circ 
\end{bmatrix}=\begin{bmatrix}
0.5681 & -0.5681 & 0.5681 
\end{bmatrix}$

Since we want to compute double the angle, the new angle is $60.8^\circ$. Next we compute $\cos30.4^\circ=0.862$ and $\sin30.4^\circ=0.506$. So, we can construct the new quaternion $\begin{bmatrix}
\cos30.4^\circ & \sin30.4^\circ\cdot0.5681 & \sin30.4^\circ\cdot-0.5681 & \sin30.4^\circ\cdot0.5681
\end{bmatrix}=\begin{bmatrix}
	0.862 & 0.288 & -0.288 & 0.288
\end{bmatrix}$.

\subsection{Exercise 8}

See \path{exercise08.cpp} for the computations of item b and c. Too much effort to compute it manually.

\begin{enumerate}[a.]
	\item % a
	$(0.233\cdot-0.752)+(0.06\cdot0.286)+(-0.257\cdot0.374)+(-0.935\cdot0.459)=-0.6833$.
	\item % b
	$\begin{bmatrix}
		0.333 & \begin{pmatrix}
			0.253 & -0.015 & 0.906
		\end{pmatrix}
	\end{bmatrix}$
	\item % c
	$\begin{bmatrix}
		-0.683 & \begin{pmatrix}
			0.343 & -0.401 & -0.500
		\end{pmatrix}
	\end{bmatrix}$
\end{enumerate}

\subsection{Exercise 9}

Show that $\|q_1q_2\|=\|q_1\|\|q_2\|$. Good luck, he he. :]

\subsection{Exercise 10}

Really?

\end{document}
