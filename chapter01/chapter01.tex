\documentclass[11pt]{article}

\author{wilricknl}
\title{3D Math Primer - Solutions chapter 1}

\usepackage{amsmath}
\usepackage{enumerate}

\begin{document}

\maketitle

\section{Solutions chapter 1}

\subsection{Exercise 1}

\begin{enumerate}[a.]
\item $(-2.5, 2.5)$
\item $(1, 2)$
\item $(2.5, 2)$
\item $(-1, 1)$
\item $(0, 0)$
\item $(2, -0.5)$
\item $(-0.5, 1.5)$
\item $(0, -2)$
\item $(-3, -2)$
\end{enumerate}

\subsection{Exercise 2}

\begin{enumerate}[a.]
\item (1, 2, 4)
\item (-3, -3, -5)
\item (-3, 6, 2.5)
\item (3, 0, -1)
\item (0, 0, 0)
\item (0, 0, 3)
\item (-3.5, 4, 0)
\item (5, -5, -1.5)
\item (4, 1, 5)
\end{enumerate}

\subsection{Exercise 3}

Pointless to write down. :]


\subsection{Exercise 4}

\begin{enumerate}[a.]
\item Right handed.
\item Swap $y$ and $z$.
\item Swap $y$ and $z$.
\end{enumerate}

\subsection{Exercise 5}

\begin{enumerate}[a.]
\item Right handed.
\item Swap $x$ with $z_{convention}$, $y$ with $x_{convention}$ and $z$ with $-y_{convention}$.
\item Swap $x_{convention}$ with $y$, $z_{convention}$ with $x$ and $y_{convention}$ with $-z$.
\end{enumerate}

\subsection{Exercise 6}

\begin{enumerate}[a.]
\item Clock wise.
\item Counter clock wise.
\item Counter clock wise.
\item Clock wise.
\end{enumerate}

\subsection{Exercise 7}

\begin{enumerate}[a.]
\item $15$
\item $30$
\item $2*4*6*8*10$
\item $120 * 7^5$
\item $49 * 100 + 100 + 50 = 5050$
\end{enumerate}

\subsection{Exercise 8}

\begin{enumerate}[a.]
\item $\frac{30 \pi}{180}=\frac{\pi}{6}$
\item $\frac{-45\pi}{180}=\frac{-\pi}{4}$
\item $\frac{60\pi}{180}$
\item $\frac{\pi}{2}$
\item $-\pi$
\item $\frac{225\pi}{180}=\frac{5\pi}{4}$
\item $\frac{3\pi}{2}$
\item $\frac{167.5\pi}{180}$
\item $\frac{527\pi}{180}=2\pi+\frac{167\pi}{180}$
\item $-6\pi$
\end{enumerate}

\subsection{Exercise 9}

\begin{enumerate}[a.]
\item $\frac{-\pi}{6}/\pi*180=\frac{-180}{6}=-30^\circ$
\item $\frac{2\pi}{3}=\frac{2*180}{3}=\frac{360}{3}=120^\circ$ 
\item $(\frac{3\pi}{2})/\pi*180=\frac{540}{2}=270^\circ$ 
\item $(\frac{-4}{3}*180)=\frac{-720}{3}=240^\circ$ 
\item $2\pi=360^\circ$
\item $1^\circ$
\item $10^\circ$
\item $1800^\circ$
\item $\frac{180}{5}=36^\circ$
\end{enumerate}

\subsection{Exercise 10}

\begin{enumerate}[1.]
\item He should define the triangle being a right triangle.
\item Also, it's not any two sides, but the triangle's two legs.
\end{enumerate}

\subsection{Exercise 11}

\begin{enumerate}[a.]
\item $$\frac{\sin(a)}{\csc(a)} + \frac{\cos(a)}{\sec(a)} =$$
$$\frac{\sin(a)}{\frac{1}{\sin(a)}} + \frac{\cos(a)}{\frac{1}{\cos(a)}} =$$
$$\sin(a)\sin(a) + \cos(a)\cos(a) =$$
$$\sin^2(a) + \cos^2(a) = 1$$
\item $$\frac{\sec^2(b)-1}{\sec^2(b)} =$$
$$\frac{\frac{1}{\cos^2(b)}-1}{\frac{1}{\cos^2(b)}} =$$
$$\cos^2(b)(\frac{1}{\cos^2(b)}-1) =$$
$$1 - \cos^2(b) = \sin^2(b)$$
\item It's already a Pythagorean identity.
\item $$\cos(c)(\tan(c)+\cot(c)) =$$
$$\cos(c)(\frac{\sin(c)}{\cos(c)} + \frac{\cos(c)}{\sin(c)})=$$
$$\sin(c) + \frac{\cos^2(c)}{\sin(c)}=$$
$$\frac{\sin^2(c)}{\sin(c)} + \frac{\cos^2(c)}{\sin(c)}=$$
$$\frac{\sin^2(c) + \cos^2(c)}{\sin(c)} = \frac{1}{\sin(c)}$$
\end{enumerate}

\end{document}
