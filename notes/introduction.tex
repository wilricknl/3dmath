\section{Introduction}

This document is a reference of the things most interesting to me from 3D math primer. This document is by no means a tutorial, so whenever something is unclear: step up your game and figure it out.

\subsection{Conventions}

\begin{itemize}
\item A left handed coordinate space is assumed unless specified otherwise.
\item \textit{Scalar} variables are represented by lowercase Roman or Greek letters in italics: $a$, $b$, $x$, $y$, $z$, $\theta$,
  $\alpha$, $\omega$, $\gamma$.
\item
  \textit{Vector} variables of any dimension are represented by lowercase
  letters in boldface: $\mathbf{a}$, $\mathbf{b}$, $\mathbf{u}$, $\mathbf{v}$, $\mathbf{q}$, $\mathbf{r}$.
\item \textit{Matrix} variables are represented using uppercase letters in
  boldface: $\mathbf{A}$, $\mathbf{B}$, $\mathbf{M}$, $\mathbf{R}$.
 \item \textit{Quaternion} variables are represented by lowercase letters in boldface: $\textbf{q}$. The difference between quaternions and vectors will be clear by the context.
\end{itemize}

\subsection{Math notation}

\begin{itemize}
	\item \textit{Interval notation}; often times a range of numbers is required. For this we use interval notation. I always forget about it, so I guess it's useful to write it all down:
	\begin{itemize}
		\item \textit{Open} interval: $(0,1)$ denotes all points $0 < x < 1$.
		\item \textit{Closed} interval: $[0,\pi]$ denotes all points $0 \leq x \leq \pi$.
		\item \textit{Mixed} intervals: $(-90^\circ, 90^\circ]$ denotes the points $90^\circ < x \leq 90^\circ$.
	\end{itemize}
\end{itemize}
