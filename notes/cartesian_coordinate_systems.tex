\section{Cartesian Coordinate Systems}

\subsection{Coordinate system handedness}

In 3D space there are two types of coordinate systems: the so-called left and right handed coordinate systems. The way to recreate these systems is to make your index fingers pointer up, then your thumbs should point to each other, while your third finger points away from you. Just like in algebra class your index finger represents the $y$-axis, your thumbs the $x$-axis, while your third finger represents the $z$-axis.

\begin{figure}[H]
\centering
    \begin{tikzpicture}
		\draw [red,-{Stealth}] (-4,0)--(-2,0) node[right] (xleft) {x};        
		\draw [green,-{Stealth}] (-4,0)--(-4,2) node[above]{y};       
		\draw [blue,-{Stealth}] (-4,0)--(-2.5,1) node[above] (zleft) {z};  
		\draw[->] (zleft) to[bend left] node [right] {y} (xleft);
		\node at (-4,-0.5) {Left handed};      
        
		\draw [red,{Stealth}-] (2,0) node[left] (xright) {x} -- (4,0);        
		\draw [green,-{Stealth}] (4,0)--(4,2) node[above]{y};       
		\draw [blue,-{Stealth}] (4,0)--(2.5,1) node[above] (zright) {z};
		\draw[->] (zright) to[bend right] node [left] {y} (xright);  
		\node at (3,-0.5) {Right handed};
		                
    \end{tikzpicture}
\caption{Coordinate system handedness}
\label{fig:coordinate-system-handedness}
\end{figure}

Rotation in a left handed in a left handed system is clock wise, while positive in a right handed system is counter clock wise. In figure  \ref{fig:coordinate-system-handedness} the $y$-rotation of both  coordinate systems is visualized with the black arrow. A trick to remember is to put your thumb up and see how your fingers curl around your palm.

\begin{quote}
\emph{Within these notes I follow the book and use the left handed system as visualized in figure \ref{fig:coordinate-system-handedness}.}
\end{quote}

\subsection{Trigonometry}

\subsubsection{Degrees and radians}

\[
\begin{array}{rll}
{1\ {rad} =} & {\left( 180/\pi \right)^{o}} & {\approx 57.29578^{o},} \\
{1^{o} =} & {\left( \pi/180 \right)\ {rad}} & {\approx 0.01745329\ {rad}.} \\
\end{array}
\]


\subsubsection{Functions}

\begin{align*}
\cos\theta &=x, &\sin\theta &=y, \\
\sec\theta &=\frac{1}{\cos\theta}, &\tan\theta &=\frac{\sin\theta}{\cos\theta}, \\
\csc\theta &=\frac{1}{\sin\theta}, &\cot\theta &=\frac{1}{\tan\theta}=\frac{\cos\theta}{\sin\theta}.
\end{align*}

\subsubsection{SOH CAH TOA}

The primary function are defined by the ratios of a \emph{right} triangle. Note that when the angle is obtuse, i.e. $90^\circ < \theta < 180^\circ$, the ratios do not work.

\begin{figure}[H]
\centering
    \includegraphics{01_sohcahtoa}
\caption{SOH CAH TOA visualisation}
\label{fig:soh-cah-toa-visualization}
\end{figure}

\[\begin{matrix}
{\cos\theta} & {= \frac{\mathit{a}\mathit{d}\mathit{j}}{\mathit{h}\mathit{y}\mathit{p}},} & {\sin\theta} & {= \frac{\mathit{o}\mathit{p}\mathit{p}}{\mathit{h}\mathit{y}\mathit{p}},} & {\tan\theta} & {= \frac{\mathit{o}\mathit{p}\mathit{p}}{\mathit{a}\mathit{d}\mathit{j}},} \\
 & & & & & \\
{\sec\theta} & {= \frac{\mathit{h}\mathit{y}\mathit{p}}{\mathit{a}\mathit{d}\mathit{j}},} & {\csc\theta} & {= \frac{\mathit{h}\mathit{y}\mathit{p}}{\mathit{o}\mathit{p}\mathit{p}},} & {\cot\theta} & {= \frac{\mathit{a}\mathit{d}\mathit{j}}{\mathit{o}\mathit{p}\mathit{p}}.} \\
\end{matrix}\]

The general definitions are defined as follows:
\begin{figure}[H]
\centering
    \includegraphics{01_general_trigonometry_definitions}
\caption{General trigonometry definitions}
\label{fig:general-trigonometry-definitions}
\end{figure}

\[\begin{matrix}
{\cos\theta} & {= x/r,} & {\sin\theta} & {= y/r,} & {\tan\theta} & {= y/x,} \\
 & & & & & \\
{\sec\theta} & {= r/x,} & {\csc\theta} & {= r/y,} & {\cot\theta} & {= x/y.} \\
\end{matrix}\]

\subsubsection{Identities}

\[
\begin{array}{rlrlrl}
{\sin( - \theta) =} & {- \sin\theta,} & {\cos( - \theta) =} & {\cos\theta,} & {\tan( - \theta) =} & {- \tan\theta,} \\
{\sin\left( \frac{\pi}{2} - \theta \right) =} & {\cos\theta,} & {\cos\left( \frac{\pi}{2} - \theta \right) =} & {\sin\theta,} & {\tan\left( \frac{\pi}{2} - \theta \right) =} & {\cot\theta.}
\end{array}
\]

\subsubsection{Common values}

\(\begin{matrix}
{\theta^{o}} & {\theta\ {rad}} & {\cos\theta} & {\sin\theta} & {\tan\theta} & {\sec\theta} & {\csc\theta} & {\cot\theta} \\
0 & 0 & 1 & 0 & 0 & 1 & {undef} & {undef} \\
30 & {\frac{\pi}{6} \approx 0.5236} & \frac{\sqrt{3}}{2} & \frac{1}{2} & \frac{\sqrt{3}}{3} & \frac{2\sqrt{3}}{3} & 2 & \sqrt{3} \\
45 & {\frac{\pi}{4} \approx 0.7854} & \frac{\sqrt{2}}{2} & \frac{\sqrt{2}}{2} & 1 & \sqrt{2} & \sqrt{2} & 1 \\
60 & {\frac{\pi}{3} \approx 1.0472} & \frac{1}{2} & \frac{\sqrt{3}}{2} & \sqrt{3} & 2 & \frac{2\sqrt{3}}{3} & \frac{\sqrt{3}}{3} \\
90 & {\frac{\pi}{2} \approx 1.5708} & 0 & 1 & {undef} & {undef} & 1 & 0 \\
120 & {\frac{2\pi}{3} \approx 2.0944} & {- \frac{1}{2}} & \frac{\sqrt{3}}{2} & {- \sqrt{3}} & {- 2} & \frac{2\sqrt{3}}{3} & {- \frac{\sqrt{3}}{3}} \\
135 & {\frac{3\pi}{4} \approx 2.3562} & {- \frac{\sqrt{2}}{2}} & \frac{\sqrt{2}}{2} & {- 1} & {- \sqrt{2}} & \sqrt{2} & {- 1} \\
150 & {\frac{5\pi}{6} \approx 2.6180} & {- \frac{\sqrt{3}}{2}} & \frac{1}{2} & {- \frac{\sqrt{3}}{3}} & {- \frac{2\sqrt{3}}{3}} & 2 & {- \sqrt{3}} \\
180 & {\pi \approx 3.1416} & {- 1} & 0 & 0 & {- 1} & {undef} & {undef} \\
210 & {\frac{7\pi}{6} \approx 3.6652} & {- \frac{\sqrt{3}}{2}} & {- \frac{1}{2}} & \frac{\sqrt{3}}{3} & {- \frac{2\sqrt{3}}{3}} & {- 2} & {- \sqrt{3}} \\
225 & {\frac{5\pi}{4} \approx 3.9270} & {- \frac{\sqrt{2}}{2}} & {- \frac{\sqrt{2}}{2}} & 1 & {- \sqrt{2}} & {- \sqrt{2}} & {- 1} \\
240 & {\frac{4\pi}{3} \approx 4.1888} & {- \frac{1}{2}} & {- \frac{\sqrt{3}}{2}} & \sqrt{3} & {- 2} & {- \frac{2\sqrt{3}}{3}} & {- \frac{\sqrt{3}}{3}} \\
270 & {\frac{3\pi}{2} \approx 4.7124} & 0 & {- 1} & {undef} & {undef} & {- 1} & 0 \\
300 & {\frac{5\pi}{3} \approx 5.2360} & \frac{1}{2} & {- \frac{\sqrt{3}}{2}} & {- \sqrt{3}} & 2 & {- \frac{2\sqrt{3}}{3}} & {- \frac{\sqrt{3}}{3}} \\
315 & {\frac{7\pi}{4} \approx 5.4978} & \frac{\sqrt{2}}{2} & {- \frac{\sqrt{2}}{2}} & {- 1} & \sqrt{2} & {- \sqrt{2}} & {- 1} \\
330 & {\frac{11\pi}{6} \approx 5.7596} & \frac{\sqrt{3}}{2} & {- \frac{1}{2}} & {- \frac{\sqrt{3}}{3}} & \frac{2\sqrt{3}}{3} & {- 2} & {- \sqrt{3}} \\
360 & {2\pi \approx 6.2832} & 1 & 0 & 0 & 1 & {undef} & {undef} \\
\end{matrix}\)

\subsubsection{Sum and difference identities}

\[\begin{matrix}
{\sin(a + b)} & {= \sin a\cos b + \cos a\sin b,} \\
{\sin(a - b)} & {= \sin a\cos b - \cos a\sin b,} \\
{\cos(a + b)} & {= \cos a\cos b - \sin a\sin b,} \\
{\cos(a - b)} & {= \cos a\cos b + \sin a\sin b,} \\
{\tan(a + b)} & {= \frac{\tan a + \tan b}{1 - \tan a\tan b},} \\
{\tan(a - b)} & {= \frac{\tan a - \tan b}{1 + \tan a\tan b}.} \\
\end{matrix}\]

\subsubsection{Pythagorean identities}

\[\begin{matrix}
{\sin^{2}\theta + \cos^{2}\theta} & {= 1,} & {1 + \tan^{2}\theta} & {= \sec^{2}\theta,} & {1 + \cot^{2}\theta} & {= \csc^{2}\theta.} \\
\end{matrix}\]

\subsubsection{Double angle identities}

\[\begin{matrix}
{\sin 2\theta} & {= 2\sin\theta\cos\theta,} \\
{\cos 2\theta} & {= \cos^{2}\theta - \sin^{2}\theta = 2\cos^{2}\theta - 1 = 1 - 2\sin^{2}\theta,} \\
{\tan 2\theta} & {= \frac{2\tan\theta}{1 - \tan^{2}\theta}.} \\
\end{matrix}\]

\subsubsection{Law of sines and cosines}\label{sin_cos_laws}

\begin{figure}[H]
\centering
    \includegraphics{01_laws}
\caption{Laws of sines and cosines}
\label{fig:laws-of-sines-and-cosines}
\end{figure}

$$\frac{\sin A}{a}=\frac{\sin B}{b}=\frac{\sin C}{c}$$

\begin{align*}
a^2 &= b^2 + c^2 - 2bc\cos A, \\
b^2 &= a^2 + c^2 - 2ac\cos B, \\
c^2 &= a^2 + b^2 - 2ab\cos C.
\end{align*}
