\documentclass[11pt]{article}

\author{wilricknl}
\title{3D Math Primer - Solutions chapter 4}

\usepackage{amsmath}
\usepackage{amssymb}
\usepackage{enumerate}
\usepackage{graphicx}

\begin{document}

\maketitle

\section{Solutions chapter 4}

\subsection{Exercise 1}

\begin{itemize}
	\item $\textbf{A}=4 \times 3$
	\item $\textbf{B}=3 \times 3$ (square and diagonal)
	\item $\textbf{C}=2 \times 2$ (square)
	\item $\textbf{D}=5 \times 2$
	\item $\textbf{E}=1 \times 3$
	\item $\textbf{F}=4 \times 1$
	\item $\textbf{G}=1 \times 4$
	\item $\textbf{H}=3 \times 1$
\end{itemize}

\subsection{Exercise 2}

\begin{itemize}
	\item $\textbf{A}^\intercal=\begin{bmatrix}
		13 & 12 & -3 & 10 \\
		4 & 0 & -1 & -2 \\
		-8 & 6 & 5 & 5
	\end{bmatrix}$
	\item $\textbf{B}^\intercal=\textbf{B}^{-1}=\textbf{B}$
	\item $\textbf{C}^\intercal=\begin{bmatrix}
		15 & -7 \\
		8 & 3
	\end{bmatrix}$
	\item $\textbf{D}^\intercal=\begin{bmatrix}
		a & b & c & d & f \\
		g & h & i & j & k
	\end{bmatrix}$
	\item $\textbf{E}^\intercal=\begin{bmatrix}
		0 \\ 1 \\ 3
	\end{bmatrix}$
	\item $\textbf{F}^\intercal=\begin{bmatrix}
		x & y & z & w
	\end{bmatrix}$
	\item $\textbf{G}^\intercal=\begin{bmatrix}
		10 \\ 20 \\ 30 \\ 1
	\end{bmatrix}$
	\item $\textbf{H}^\intercal=\begin{bmatrix}
		\alpha & \beta & \gamma
	\end{bmatrix}$
\end{itemize}

\subsection{Exercise 3}

\begin{enumerate}[1.)]
	\item $\textbf{AB}=4 \times 3$
	\item $\textbf{AH}=4 \times 1$
	\item $\textbf{BB}=3 \times 3$
	\item $\textbf{BH}=3 \times 1$
	\item $\textbf{CC}=2 \times 2$
	\item $\textbf{DC}=5 \times 2$
	\item $\textbf{EB}=1 \times 3$
	\item $\textbf{EH}=1 \times 1$
	\item $\textbf{FE}=4 \times 3$
	\item $\textbf{FG}=4 \times 4$
	\item $\textbf{GA}=1 \times 3$
	\item $\textbf{GF}=1 \times 1$
	\item $\textbf{HE}=3 \times 3$
	\item $\textbf{HG}=3 \times 4$
\end{enumerate}

\subsection{Exercise 4}

\begin{enumerate}[a.]
	\item % a
	$\begin{bmatrix}
		1 & -2 \\
		5 & 0
	\end{bmatrix}
	\begin{bmatrix}
		-3 & 7 \\
		4 & \frac{1}{3}
	\end{bmatrix}=
	\begin{bmatrix}
		-3+-8 & 7-\frac{2}{3} \\
		-15 & 35
	\end{bmatrix}=
	\begin{bmatrix}
		-11 & 6\frac{1}{3} \\
		-15 & 35
	\end{bmatrix}$
	\item % b
	Not possible.
	\item % c
	$\begin{bmatrix}
		3 & -1 & 4
	\end{bmatrix}
	\begin{bmatrix}
		-2 & 0 & 3 \\
		5 & 7 & -6 \\
		1 & -4 & 2
	\end{bmatrix}=
	\begin{bmatrix}
		-6-5+4 & -7-16 & 9+6+8
	\end{bmatrix}=
	\begin{bmatrix}
		-7 & -23 & 23
	\end{bmatrix}
	$
	\item % d
	$\begin{bmatrix}
		x & y & z & w
	\end{bmatrix}$
	\item % e
	Not possible.
	\item % f
	$\begin{bmatrix}
		m_{11} & m_{12} \\
		m_{21} & m_{22}
	\end{bmatrix}$
	\item % g
	$\begin{bmatrix}
		3 & 3
	\end{bmatrix}
	\begin{bmatrix}
		6 & -7 \\
		-4 & 5
	\end{bmatrix}=
	\begin{bmatrix}
		18-12 & -21+15
	\end{bmatrix}=
	\begin{bmatrix}
		6 & -6
	\end{bmatrix}			
	$
	\item % h
	Not possible.
\end{enumerate}

\subsection{Exercise 5}

\begin{enumerate}[a.]
	\item % a
	The same, because it's the identity matrix.
	\item % b
	Not the same, because the matrix is not symmetrical.
	
	$\begin{bmatrix}
	 5 & -1 & 2
	\end{bmatrix}
	\begin{bmatrix}
		2 & 5 & -3 \\
		1 & 7 & 1 \\
		-2 & -1 & 4
	\end{bmatrix}=
	\begin{bmatrix}
		5 & 16 & -8
	\end{bmatrix}$
	
	$\begin{bmatrix}
		2 & 5 & -3 \\
		1 & 7 & 1 \\
		-2 & -1 & 4
	\end{bmatrix}
	\begin{bmatrix}
		5 \\ -1 \\ 2
	\end{bmatrix}=
	\begin{bmatrix}
		-1 \\ 0 \\ -1
	\end{bmatrix}$
	
	\item % c
	The same, the matrix is symmetrical.
	
	$\begin{bmatrix}
		5 & -1 & 2
	\end{bmatrix}
	\begin{bmatrix}
		1 & 7 & 2 \\
		7 & 0 & -3 \\
		2 & -3 & -1
	\end{bmatrix}=
	\begin{bmatrix}
		2 & 29 & 11
	\end{bmatrix}$
	
	$\begin{bmatrix}
		1 & 7 & 2 \\
		7 & 0 & -3 \\
		2 & -3 & -1
	\end{bmatrix}
	\begin{bmatrix}
		5 \\ -1 \\ 2
	\end{bmatrix}=
	\begin{bmatrix}
		2 \\ 29 \\ 11
	\end{bmatrix}$	
	
	\item % d
	Not the same, there is no symmetry once more.

	$\begin{bmatrix}
		5 & -1 & 2
	\end{bmatrix}
	\begin{bmatrix}
		1 & -4 & 3 \\
		4 & 0 & -1 \\
		-3 & 1 & 0
	\end{bmatrix}=
	\begin{bmatrix}
		-10 & -18 & 16
	\end{bmatrix}$
	
	$\begin{bmatrix}
		1 & -4 & 3 \\
		4 & 0 & -1 \\
		-3 & 1 & 0
	\end{bmatrix}
	\begin{bmatrix}
		5 \\ -1 \\ 2
	\end{bmatrix}=
	\begin{bmatrix}
		10 \\ 18 \\ -16
	\end{bmatrix}$	
\end{enumerate}

\subsection{Exercise 6}

\begin{enumerate}[a.]
	\item $((\textbf{A}^\intercal)^\intercal)^\intercal=\textbf{A}^\intercal$
	\item $(\textbf{BA}^\intercal)^\intercal\textbf{CD}^\intercal=\textbf{AB}^\intercal\textbf{CD}^\intercal$
	\item $((\textbf{D}^\intercal\textbf{C}^\intercal)(\textbf{AB})^\intercal)^\intercal=((\textbf{D}^\intercal\textbf{C}^\intercal)(\textbf{B}^\intercal\textbf{A}^\intercal))^\intercal=(\textbf{B}^\intercal\textbf{A}^\intercal)^\intercal(\textbf{D}^\intercal\textbf{C}^\intercal)^\intercal=\textbf{ABCD}$
	\item $((\textbf{AB})^\intercal(\textbf{CDE})^\intercal)^\intercal=\textbf{CDEAB}$
\end{enumerate}

\subsection{Exercise 7}

\begin{enumerate}[a.]
	\item $90^\circ$ rotated clockwise.
	\item $45^\circ$ rotated counterclockwise.
	\item Uniformly scaled.
	\item Non-uniformly scaled.
	\item Flipped $x$-axis.
	\item Rotated $90^\circ$ and scaled uniformly.
\end{enumerate}

\subsection{Exercise 8}

$\textbf{a}\times\textbf{b}=\begin{bmatrix}
	a_x \\ a_y \\ a_z
\end{bmatrix}\times\begin{bmatrix}
	b_x \\ b_y \\ b_z
\end{bmatrix}=
\begin{bmatrix}
	a_yb_z-a_zb_y \\ a_zb_x-a_xb_z \\ a_xb_y-a_yb_x
\end{bmatrix}$ \\
$\textbf{aM}$ gives the following equations:

\begin{itemize}
	\item $a_xm_{11}+a_ym_{21}+a_zm_{31}$ which has to be equal to $a_yb_z-a_zb_y$, so we can deduce $m_{11}=0$, $m_{21}=b_z$ and $m_{31}=-b_y$.
	\item $a_xm_{12}+a_ym_{22}+a_zm_{32}$ which has to be equal to $a_zb_x-a_xb_z$, so we can deduce $m_{12}=-b_z$, $m_{22}=0$ and $m_{32}=b_x$.
	\item $a_xm_{13}+a_ym_{23}+a_zm_{33}$ which has to be equal to $a_xb_y-a_yb_x$, so we can deduce $m_{13}=b_y$, $m_{23}=-b_x$ and $m_{33}=0$.
\end{itemize}

The combination of these values results in the following matrix: $\begin{bmatrix}
	0 & -b_z & b_y \\
	b_z & 0 & -b_x \\
	-b_y & b_x & 0
\end{bmatrix}$

\subsection{Exercise 9}

\begin{enumerate}[a.]
	\item 3
	\item 1
	\item 4
	\item 2
\end{enumerate}

\subsection{Exercise 10}

$
\textbf{M}=\begin{bmatrix}
	1 & 0 & 0 & 0 & 0 & 0 & 0 & 0 & 0 & 0 \\
	-1 & 1 & 0 & 0 & 0 & 0 & 0 & 0 & 0 & 0 \\
	0 & -1 & 1 & 0 & 0 & 0 & 0 & 0 & 0 & 0 \\
	0 & 0 & -1 & 1 & 0 & 0 & 0 & 0 & 0 & 0 \\
	0 & 0 & 0 & -1 & 1 & 0 & 0 & 0 & 0 & 0 \\
	0 & 0 & 0 & 0 & -1 & 1 & 0 & 0 & 0 & 0 \\
	0 & 0 & 0 & 0 & 0 & -1 & 1 & 0 & 0 & 0 \\
	0 & 0 & 0 & 0 & 0 & 0 & -1 & 1 & 0 & 0 \\
	0 & 0 & 0 & 0 & 0 & 0 & 0 & -1 & 1 & 0\\
	0 & 0 & 0 & 0 & 0 & 0 & 0 & 0 & -1 & 1
\end{bmatrix}
$

\subsection{Exercise 11}

$
\textbf{N}=\begin{bmatrix}
	1 & 0 & 0 & 0 & 0 & 0 & 0 & 0 & 0 & 0 \\
	1 & 1 & 0 & 0 & 0 & 0 & 0 & 0 & 0 & 0 \\
	1 & 1 & 1 & 0 & 0 & 0 & 0 & 0 & 0 & 0 \\
	1 & 1 & 1 & 1 & 0 & 0 & 0 & 0 & 0 & 0 \\
	1 & 1 & 1 & 1 & 1 & 0 & 0 & 0 & 0 & 0 \\
	1 & 1 & 1 & 1 & 1 & 1 & 0 & 0 & 0 & 0 \\
	1 & 1 & 1 & 1 & 1 & 1 & 1 & 0 & 0 & 0 \\
	1 & 1 & 1 & 1 & 1 & 1 & 1 & 1 & 0 & 0 \\
	1 & 1 & 1 & 1 & 1 & 1 & 1 & 1 & 1 & 0\\
	1 & 1 & 1 & 1 & 1 & 1 & 1 & 1 & 1 & 1
\end{bmatrix}
$

\subsection{Exercise 12}

\begin{itemize}
	\item Integration and differentiation cancel each other out, so $\textbf{M}\textbf{N}=\textbf{N}\textbf{M}=\textbf{I}$.
	\item $\textbf{M}\textbf{N}=\textbf{N}\textbf{M}=\begin{bmatrix}
	1 & 0 & 0 & 0 & 0 & 0 & 0 & 0 & 0 & 0 \\
	0 & 1 & 0 & 0 & 0 & 0 & 0 & 0 & 0 & 0 \\
	0 & 0 & 1 & 0 & 0 & 0 & 0 & 0 & 0 & 0 \\
	0 & 0 & 0 & 1 & 0 & 0 & 0 & 0 & 0 & 0 \\
	0 & 0 & 0 & 0 & 1 & 0 & 0 & 0 & 0 & 0 \\
	0 & 0 & 0 & 0 & 0 & 1 & 0 & 0 & 0 & 0 \\
	0 & 0 & 0 & 0 & 0 & 0 & 1 & 0 & 0 & 0 \\
	0 & 0 & 0 & 0 & 0 & 0 & 0 & 1 & 0 & 0 \\
	0 & 0 & 0 & 0 & 0 & 0 & 0 & 0 & 1 & 0\\
	0 & 0 & 0 & 0 & 0 & 0 & 0 & 0 & 0 & 1
\end{bmatrix}$
\end{itemize}

\end{document}