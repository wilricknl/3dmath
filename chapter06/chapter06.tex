\documentclass[11pt]{article}

\author{wilricknl}
\title{3D Math Primer - Solutions chapter 6}

\usepackage{amsmath}
\usepackage{amssymb}
\usepackage{enumerate}
\usepackage{graphicx}

\begin{document}

\maketitle

\section{Solutions chapter 6}

\subsection{Exercise 1}

$3 \cdot 4 - 1 \cdot -2 = 14$

\subsection{Exercise 2}

\begin{itemize}
	\item Determinant:
$\left(\begin{bmatrix}
3 \\ -2 \\ 0
\end{bmatrix}\times
\begin{bmatrix}
1 \\ 4 \\ 0
\end{bmatrix}
\right)\cdot
\begin{bmatrix}
0 \\ 0 \\ 2
\end{bmatrix}=
\begin{bmatrix}
0 \\ 0 \\ 14
\end{bmatrix}\cdot
\begin{bmatrix}
0 \\ 0 \\ 2
\end{bmatrix}=28
$
	\item Adjoint:
$
\begin{bmatrix}
+8 & -2 & +0 \\
--4=4 & +6 & -0 \\
+0 & -0 & +14
\end{bmatrix}^\intercal=
\begin{bmatrix}
8 & 4 & 0 \\
-2 & 6 & 0 \\
0 & 0 & 14
\end{bmatrix}
$
	\item Inverse (adjoint divided by determinant):
$
\begin{bmatrix}
8 & 4 & 0 \\
-2 & 6 & 0 \\
0 & 0 & 14
\end{bmatrix}/14=
\begin{bmatrix}
\frac{2}{7} & \frac{1}{7} & 0 \\
\frac{-1}{14} & \frac{3}{14} & 0 \\
0 & 0 & \frac{1}{2}
\end{bmatrix}
$

\end{itemize}

\subsection{Exercise 3}

$
\begin{bmatrix}
-0.1495 & -0.1986 & -0.9685 \\
-0.8256 & 0.5640 & 0.0117 \\
-0.5439 & -0.8015 & 0.2484
\end{bmatrix}
\begin{bmatrix}
-0.1495 & -0.1986 & -0.9685 \\
-0.8256 & 0.5640 & 0.0117 \\
-0.5439 & -0.8015 & 0.2484
\end{bmatrix}^\intercal=\textbf{I}
$

\subsection{Exercise 4}

You can simply take the transpose, since we proved in the previous exercise that the matrix is orthogonal.

\subsection{Exercise 5}

Remember that we computed in the previous exercise that the inverse and the transpose of the first matrix are the same.

$$
\left(
\begin{bmatrix}
-0.1495 & -0.1986 & -0.9685 & 0 \\
-0.8256 & 0.5640 & 0.0117 & 0 \\
-0.5439 & -0.8015 & 0.2484 & 0 \\
0 & 0 & 0 & 1
\end{bmatrix}
\begin{bmatrix}
1 & 0 & 0 & 0 \\
0 & 1 & 0 & 0 \\
0 & 0 & 1 & 0 \\
1.7928 & -5.3116 & 8.0151 & 1 
\end{bmatrix}
\right)^{-1}=
$$

$$
\left(
\begin{bmatrix}
1 & 0 & 0 & 0 \\
0 & 1 & 0 & 0 \\
0 & 0 & 1 & 0 \\
1.7928 & -5.3116 & 8.0151 & 1 
\end{bmatrix}^{-1}
\begin{bmatrix}
-0.1495 & -0.1986 & -0.9685 & 0 \\
-0.8256 & 0.5640 & 0.0117 & 0 \\
-0.5439 & -0.8015 & 0.2484 & 0 \\
0 & 0 & 0 & 1
\end{bmatrix}^\intercal
\right)=
$$

$$
\left(
\begin{bmatrix}
1 & 0 & 0 & 0 \\
0 & 1 & 0 & 0 \\
0 & 0 & 1 & 0 \\
-1.7928 & 5.3116 & -8.0151 & 1 
\end{bmatrix}
\begin{bmatrix}
-0.1495 & -0.8256 & -0.5439 & 0 \\
-0.1986 & 0.5640 & -0.8015 & 0 \\
-0.9685 & 0.0117 & 0.2484 & 0 \\
0 & 0 & 0 & 1
\end{bmatrix}
\right)=
$$

$$
\begin{bmatrix}
-0.1495 & -0.8256 & -0.5439 & 0 \\
-0.1986 & 0.5640 & -0.8015 & 0 \\
-0.9685 & 0.0117 & 0.2484 & 0 \\
6.976 & 4.382 & -5.273 & 1
\end{bmatrix}
$$

\subsection{Exercise 6}

$\textbf{T}(\begin{bmatrix}
4 & 2 & 3
\end{bmatrix})=
\begin{bmatrix}
1 & 0 & 0 & 0 \\
0 & 1 & 0 & 0 \\
0 & 0 & 1 & 0 \\
4 & 2 & 3 & 1
\end{bmatrix}
$

\subsection{Exercise 7}

First of all, let's define $20^\circ=\frac{\pi}{9}$, $\cos\frac{\pi}{9}=0.94$ and $\sin\frac{\pi}{9}=0.342$.

The rotation matrix $\textbf{R}_x(20^\circ)=
\begin{bmatrix}
1 & 0 & 0 & 0 \\
0 & 0.94 & 0.342 & 0 \\
0 & -0.342 & 0.94 & 0 \\
0 & 0 & 0 & 1
\end{bmatrix}
$

The translation matrix we already defined in the previous exercise, so all there's left to do is to actually combine them:

$
\begin{bmatrix}
1 & 0 & 0 & 0 \\
0 & 0.94 & 0.342 & 0 \\
0 & -0.342 & 0.94 & 0 \\
0 & 0 & 0 & 1
\end{bmatrix}
\begin{bmatrix}
1 & 0 & 0 & 0 \\
0 & 1 & 0 & 0 \\
0 & 0 & 1 & 0 \\
4 & 2 & 3 & 1
\end{bmatrix}
=
\begin{bmatrix}
1 & 0 & 0 & 0 \\
0 & 0.94 & 0.342 & 0 \\
0 & -0.342 & 0.94 & 0 \\
4 & 2 & 3 & 1
\end{bmatrix}
$

\subsection{Exercise 8}

This time we reuse the matrices, but multiply them in reverse order. In this case $m_{32}$ and $m_{33}$ require some more complex computing: \\
$
m_{32} = 2*0.940 + 3*-0.342 = 0.854 \\
m_{33} = 2*0.342 + 3*0.940 = 3.504
$

The full equation:
$
\begin{bmatrix}
1 & 0 & 0 & 0 \\
0 & 1 & 0 & 0 \\
0 & 0 & 1 & 0 \\
4 & 2 & 3 & 1
\end{bmatrix}
\begin{bmatrix}
1 & 0 & 0 & 0 \\
0 & 0.94 & 0.342 & 0 \\
0 & -0.342 & 0.94 & 0 \\
0 & 0 & 0 & 1
\end{bmatrix}=
\begin{bmatrix}
1 & 0 & 0 & 0 \\
0 & 0.94 & 0.342 & 0 \\
0 & -0.342 & 0.94 & 0 \\
4 & 0.854 & 3.504 & 1
\end{bmatrix}
$

\subsection{Exercise 9}

$
\begin{bmatrix}
1 & 0 & 0 & \frac{1}{5} \\
0 & 1 & 0 & 0 \\
0 & 0 & 1 & 0 \\
0 & 0 & 0 & 0
\end{bmatrix}
$

\subsection{Exercise 10}

$\begin{bmatrix}
105 & -243 & 89 & 1
\end{bmatrix}
\begin{bmatrix}
1 & 0 & 0 & \frac{1}{5} \\
0 & 1 & 0 & 0 \\
0 & 0 & 1 & 0 \\
0 & 0 & 0 & 0
\end{bmatrix}=
\begin{bmatrix}
105 & -243 & 89 & 21
\end{bmatrix}$

$
\begin{bmatrix}
105 & -243 & 89 & 21
\end{bmatrix}\to_{3D}
\begin{bmatrix}
\frac{105}{21} & \frac{-243}{21} & \frac{89}{21}
\end{bmatrix}
$

\end{document}
